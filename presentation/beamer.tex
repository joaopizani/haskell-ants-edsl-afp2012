\documentclass{beamer}
\usetheme[style=plain]{uu}

%% unicode support, for text and code :)
\usepackage[utf8x]{inputenc}
\usepackage{ucs}
\usepackage{autofe}

\usepackage{amsmath}
\usepackage{stmaryrd}
\usepackage{amssymb}
\usepackage{graphicx}
\usepackage{hyperref}
\usepackage{color}


%% imports, colors and other definitions for source code
\usepackage{listingsutf8}

\usepackage{color}
\definecolor{dkgreen}{rgb}{0,0.6,0}
\definecolor{gray}{rgb}{0.5,0.5,0.5}
\definecolor{mauve}{rgb}{0.58,0,0.82}

\lstset{
    language=Haskell,
    extendedchars=\true,
    basicstyle=\footnotesize,
    numbers=left,
    numberstyle=\tiny\color{gray},
    stepnumber=1,
    numbersep=5pt,
    backgroundcolor=\color{white},
    showspaces=false,
    showtabs=false,
    tabsize=4,
    captionpos=b,
    breakatwhitespace=false,
    title=\lstname,
    keywordstyle=\color{blue},
    commentstyle=\color{dkgreen},
    stringstyle=\color{mauve},
    morekeywords={*,...}
}

%% Metainformation
\title[Ants]{AntGen EDSL}

\date{\today}

\author[Alpuim, Binsbergen, Pizani Flor]
{
    J.~Alpuim
    \and L.T. van~Binsbergen
    \and J.P.~Pizani Flor
}

\institute[Utrecht University]
{
    Department of Information and Computing Sciences,
    Utrecht University
}


%% The document itself
\begin{document}

    \begin{frame}
        \titlepage
    \end{frame}

    \begin{frame}
        \frametitle{Table of Contents}
        \tableofcontents
    \end{frame}

    \section{Application: Top-level strategy}

    \subsection{How it works}
    \begin{frame}
    \frametitle{How it works}

	   Our main strategy consists of
	   \begin{itemize}
	   \item Draw ``Highways'' from the corners of the nest;  
	   \item Look for food using a ``ricochet'' movement;
	   \item Use ``Highways'' to come back to the nest.
	   \end{itemize} 

    \end{frame}
    
    \subsection{How it should work}
    \begin{frame}
    \frametitle{How it should work}

	    \begin{itemize}
	    \item Protecting corners of the nest; 
	    \item Drawing local roads;
	    \item Avoiding congestions.
	    \end{itemize}

    \end{frame}
    

    \section{Library: AntStrategies, AntMoves}
   
    \begin{frame}
    \frametitle{Ant strategies} 
            Some examples of top-level strategies include:
            \begin{itemize}
            \item{Ricochet walk;}
            \item{Random walk;}
            \item{Follow any pheromone track;}
            \item{Draw any pheromone track...}
            \end{itemize}
    \end{frame}

    \begin{frame}
    \frametitle{Ant moves}
            Some examples of simple moves include:
            \begin{itemize}
            \item{Safe move;}
            \item{Random choices;}
            \item{Interleaving strategies...}
            \end{itemize}
    \end{frame}   


    \section{EDSL: AntImperative, AntInstruction}

    \subsection{Imperative-ish constructs}
    \begin{frame}
    \frametitle{Imperative-ish constructs}

            We have developed some imperative-like constructs, 
            in order to get a well defined flow in each program:

            \begin{itemize}
            \item Sequence;
            \item Loops (While,Forever);
            \item Conditionals (IfThen,IfThenElse,Side-Effect test,case);
            \item Boolean operators.
            \end{itemize}

    \end{frame}

    \subsection{Translation to Ant assembly}
    \begin{frame}
    \frametitle{Translation to Ant assembly}

            \begin{itemize}
            \item EDSL datatypes;
            \item Semantic functions;
            \item Boolean algebra.
            \end{itemize}
            
    \end{frame}

    \section{Composing and transforming assembly blocks}

    \subsection{Composing assembly blocks}
    \begin{frame}
    \frametitle{Composing assembly blocks}


            \begin{itemize}
            \item One composing function per imperative construct.
            \item Using supply monad to get ant states;
            \item Map AntState AntInstruction
            \begin{itemize} 
                \item with a initial and final state. 
            \end{itemize}
            \end{itemize}

    \end{frame}
   
    \subsection{Program transformations} 
    \begin{frame}
    \frametitle{Program transformations}

            We have used two functions to get a full-working ant assembly code:
            \begin{itemize}
            \item Ghost-busters;
            \item Keys to line numbers. 
            \end{itemize}

    \end{frame}


\end{document}
